\documentclass{article}

\usepackage{amsmath,geometry,amsfonts,array,makecell,enumitem,bm,esint,booktabs,multirow,mathtools,upgreek,amssymb,pgfplots,mathrsfs,nicematrix,slashed}
\usepackage[amsmath]{ntheorem}
\usepackage[hidelinks,naturalnames]{hyperref}
\usepackage[nameinlink,noabbrev]{cleveref}
\usepackage{fancyhdr}
\pagestyle{fancy}
\fancyhead[L]{\itshape\nouppercase{\leftmark}}
\fancyhead[R]{Quantum Mechanical Systems in IB}

\title{Systems in IB}
\author{Yue Wu}

\geometry{a4paper,hmargin=1.1in,vmargin=1.2in}

\setlength{\parskip}{1em}
\tolerance=1000
\emergencystretch=1em
\hyphenpenalty=1000
\exhyphenpenalty=100
\righthyphenmin=3

\pgfplotsset{compat=1.18}
\usetikzlibrary{decorations.markings}

\theoremstyle{plain}\theoremheaderfont{\normalfont\itshape}\theorembodyfont{\rmfamily}\theoremseparator{.}\newtheorem*{rem}{Remark}\newtheorem*{ex}{Example}\newtheorem*{proof}{Proof}\newtheorem*{altp}{Alternative proof}

\theoremstyle{plain}\theoremheaderfont{\normalfont\bfseries}\theorembodyfont{\rmfamily}\theoremseparator{.}\newtheorem{thm}{Theorem}[section]\newtheorem{lem}[thm]{Lemma}\newtheorem{prop}[thm]{Proposition}\newtheorem*{cor}{Corollary}\newtheorem{defn}[thm]{Definition}\newtheorem{clm}[thm]{Claim}\newtheorem{clminproof}{Claim}

\theoremstyle{break}\theoremheaderfont{\normalfont\itshape}\theorembodyfont{\rmfamily}\theoremseparator{.\medskip}\newtheorem*{proofskip}{Proof}\newtheorem*{exs}{Examples}\newtheorem*{rems}{Remarks}

\theoremstyle{break}\theoremheaderfont{\normalfont\bfseries}\theorembodyfont{\rmfamily}\theoremseparator{.\medskip}\newtheorem{lemskip}[thm]{Lemma}\newtheorem{defnskip}[thm]{Definition}\newtheorem{propskip}[thm]{Proposition}\newtheorem{thmskip}[thm]{Theorem}

\crefname{thm}{Theorem}{Theorems}\crefname{defn}{Definition}{Definitions}\crefname{lem}{Lemma}{Lemmas}\crefname{lemskip}{Lemma}{Lemmas}\crefname{cor}{Corollary}{Corollaries} \crefname{prop}{Proposition}{Propositions}\crefname{clm}{Claim}{Claims}

\setcounter{tocdepth}{2}
\setcounter{section}{0}
\numberwithin{equation}{section}

\newcommand{\qed}{\hfill\ensuremath{\Box}}
\newcommand{\unit}[1]{\ \mathrm{#1}}
\newcommand{\ii}{\mathrm{i}}
\newcommand{\ee}{\mathrm{e}}
\newcommand{\tp}{^\mathrm{T}}
\newcommand{\dd}[2][]{\mathrm{d}^{#1} #2\,}
\renewcommand{\d}[2][]{\mathrm{d}^{#1} #2}
\newcommand{\dv}[3][]{\frac{\mathrm{d}^{#1} #2}{{\mathrm{d} #3}^{#1}}}
\newcommand{\pdv}[3][]{\frac{\partial^{#1} #2}{{\partial #3}^{#1}}}
\newcommand{\bra}[1]{\left\langle #1 \right|}
\newcommand{\ket}[1]{\left| #1 \right\rangle}
\newcommand{\braket}[2]{\left\langle #1 \middle| #2 \right\rangle}
\newcommand{\mel}[3]{\left\langle #1 \middle| #2 \middle| #3 \right\rangle}
\newcommand{\redmel}[3]{\left\langle #1 \middle\| #2 \middle\| #3 \right\rangle}
\newcommand{\eval}[1]{\left\langle #1 \right\rangle}
\newcommand{\expval}[2]{\left\langle #2 \middle| #1 \middle| #2 \right\rangle}
\newcommand{\vb}[1]{\bm{\mathrm{#1}}}
\newcommand{\vu}[1]{\hat{\bm{\mathrm{#1}}}}
\newcommand{\cross}{\bm{\times}}
\newcommand{\vdot}{\bm{\cdot}}
\newcommand{\abs}[1]{\left| #1 \right|}
\newcommand{\norm}[1]{\left\| #1 \right\|}
\newcommand{\grad}{\vb{\nabla}}
\renewcommand{\div}{\vb{\nabla}\cdot}
\newcommand{\curl}{\vb{\nabla}\times}
\newcommand{\laplacian}{\nabla^2}
\renewcommand{\Re}{\operatorname{Re}}
\renewcommand{\Im}{\operatorname{Im}}
\newcommand{\hb}{\mathcal{H}}
\newcommand{\NN}{\mathbb{N}}
\newcommand{\ZZ}{\mathbb{Z}}
\newcommand{\QQ}{\mathbb{Q}}
\newcommand{\RR}{\mathbb{R}}
\newcommand{\CC}{\mathbb{C}}
\newcommand{\SO}{\mathrm{SO}}
\renewcommand{\AA}{\mathrm{A}}
\newcommand{\BB}{\mathrm{B}}
\DeclareMathOperator{\cov}{cov}
\DeclareMathOperator{\Id}{Id}
\DeclareMathOperator{\Prob}{Prob}
\DeclareMathOperator{\Dom}{Dom}
\DeclareMathOperator{\tr}{tr}
\DeclareMathOperator{\tord}{\overleftarrow{\mathcal{T}}}
\DeclareMathOperator{\trev}{\overrightarrow{\mathcal{T}}}
\newcommand{\ph}{\mathbb{P}\mathcal{H}}
\newcommand{\Sch}{^{\mathrm{S}}}
\newcommand{\Hei}{^{\mathrm{H}}}
\newcommand{\Int}{^{\mathrm{I}}}

\NiceMatrixOptions{cell-space-limits = 2pt}


\begin{document}
    Here are some important quantum mechanical systems that you should already be familiar with from Part IB Chemistry A. These serve as good reference systems that we develop further techniques upon in the C7: \textit{Further Quantum Mechanics} course.

    \section{Free Particle}
    The first system we will look at is a particle of mass \(m\) moving in a 1D space freely without external potential, described by classical Hamiltonian
    \begin{equation}
        H=\frac{p^2}{2m}\,.
    \end{equation}
    Quantisation promotes the classical momentum \(p\) into the momentum operator \(\hat{p}\) so the time-independent Schr\"{o}dinger equation \(\hat{H}\ket{\psi}=E\ket{\psi}\) reads
    \begin{equation}
        \frac{\hat{p}^2}{2m}\ket{\psi}=E\ket{\psi}\,.
    \end{equation}
    We will use the position representation so \(\hat{p}=-\ii\hbar\dv{}{x}\), and the Schr\"{o}dinger equation becomes a second order linear homogeneous equation, which can be easily solved. However, we can make the problem easier if we realise that since \(\hat{H}\) only involves \(\hat{p}\), the eigenstates of \(\hat{H}\) are also the eigenstates of \(\hat{p}\), so we can instead solve for
    \begin{equation}
        \hat{p}\ket{p}=p\ket{p}\,,
    \end{equation}
    where \(\ket{p}\) is the eigenstate of \(\hat{p}\) and \(\hat{H}\) with momentum \(p\) and energy
    \begin{equation}
        E=\frac{p^2}{2m}\,.
    \end{equation}

    This time, moving into the position representation yields a first order differential equation
    \begin{equation}
        -\ii\hbar\dv{\psi_p(x)}{x}=p\psi_p(x)\,,
    \end{equation}
    for which solution is given by
    \begin{equation}
        \psi_p(x)\equiv\braket{x}{p}=A\ee^{\ii px/\hbar}\,,
    \end{equation}
    for some normalisation constant \(A\). It is conventional to normalise the state with \(A=1/\sqrt{2\pi\hbar}\) so that \(\braket{p}{p'}=\delta(p-p')\). Notice that since we do not impose any boundary condition, \(p\) can take any real value, and the energies are non-negative. It is also common to label the state by its \textit{wavenumber} \(k=p/\hbar\).
    
    Notice that only the ground-state with zero energy is non-degenerate, with \(p=0\). Any other states with positive energy are doubly degenerate as states with momenta \(\pm p\) have the same energy. Therefore, a general expression of an energy eigenstate is
    \begin{equation}
        \ket{E}=a\ket{p}+b\ket{-p}
    \end{equation}
    for any \(a,b\in\mathbb{C}\).

    The 3D case is similar, with wavefunction
    \begin{equation}
        \braket{\vb{x}}{\vb{p}}=\frac{1}{(2\pi\hbar)^{3/2}}\ee^{\ii\vb{p}\vdot\vb{x}/\hbar}
    \end{equation}
    and energy spectrum
    \begin{equation}
        E=\frac{\vb{p}^2}{2m}
    \end{equation}
    for any \(\vb{p}\in\mathbb{R}^3\).

    \newpage
    \section{Potential Well}
    \subsection{Infinite Well}
    Next, we consider a particle moving in a infinitely deep 1D potential well of length \(L\), defined by the potential
    \begin{equation}
        V(x)=\begin{cases}
            0 & x\in[0,L] \\
            \infty & \text{otherwise.}
        \end{cases}
    \end{equation}

    Notice that inside the well (\(0\le x\le L\)), \(V=0\) so the particle is again free, with
    \begin{align}
        \braket{x}{E}&=a\ee^{\ii kx}+b\ee^{-\ii kx}\notag\\
        &=c\sin kx+ d\cos kx
    \end{align}
    and energy
    \begin{equation}
        E=\frac{\hbar^2 k^2}{2m}\,.
    \end{equation}
    
    The out-of-well situation (\(x<0\) or \(x>L\)) is a bit more subtle. The Schr\"{o}dinger equation reads
    \begin{equation}
        -\frac{\hbar^2}{2m}\dv[2]{\psi}{x}+\infty\psi=E\psi\,.
    \end{equation}
    For a state to have finite energy, we are forced to take \(\psi(x)=0\) outside the well.

    The wavefunction has to be continuous, and this puts important constraint on the wavefunction allowed inside the well. For the wavefunctions to be continuous at \(x=0\) and \(x=L\), we must have
    \begin{align}
        \psi(0)&=d=0\notag \\
        \psi(L)&=c\sin kL + d\cos kL=0\,.
    \end{align}
    These conditions give \(d=0\) and \(kL=n\pi\) for \(n\in\mathbb{Z}\). Note that \(kL=n\pi\) and \(kL=-n\pi\) are physically equivalent and \(n=0\) gives vanishing wavefunction, we reduce the range of \(n\) to \(n\in\mathbb{N}\). In conclusion, the normalised wavefunctions and energy spectrum are
    \begin{align}
        E_n&=\frac{n^2\pi^2\hbar^2}{2mL^2}\\
        \psi_n(x)&=\begin{cases}
            \sqrt{\frac{2}{L}}\sin\frac{n\pi x}{L} & x\in[0,L] \\
            0 & \text{otherwise.}
        \end{cases}
    \end{align}

    \newpage
    \section{Quantum Harmonic Oscillator}
    Here we will solve the Quantum Harmonic Oscillator in the same way as you have seen in Part IB. A much more elegant way using ladder operators will be introduced in the main text.

    Now consider a particle of mass \(m\) moving in a quadratic potential
    \begin{equation}
        V(x)=\frac{1}{2}kx^2\,.
    \end{equation}
    It proves useful to rewrite this as
    \begin{equation}
        V(x)=\frac{1}{2}m\omega^2 x^2
    \end{equation}
    so that the Hamiltonian is
    \begin{equation}
        \hat{H}=-\frac{\hbar^2}{2m}\dv[2]{}{x}+\frac{1}{2}m\omega^2 x^2\,.
    \end{equation}
    We observe that \(s=(\hbar/m\omega)^{1/2}\) has dimension \([\mathrm{L}]\), so we introduce a dimensionless quantity scaled coordinate \(q=x/s\). Under these changes of variables, the Hamiltonian becomes
    \begin{equation}
        \hat{H}=\hbar\omega\left(-\frac{1}{2}\dv[2]{}{q}+\frac{1}{2}q^2\right)\,.
    \end{equation}
    Solving the Schr\"{o}dinger equation is therefore equivalent to finding the eigenfunctions of the operator
    \begin{equation}
        \hat{L}=-\frac{1}{2}\dv[2]{}{q}+\frac{1}{2}q^2\,.
    \end{equation}

    We will try the ansatz
    \begin{equation}
        \psi(q)=p(q)\ee^{-q^2/2}\,.
    \end{equation}
    Substitution into the eigenvalue equation
    \begin{equation}
        \hat{L}\psi=\lambda\psi
    \end{equation}
    yields
    \begin{equation}
        [p''-2p'q+(2\lambda-1)p]\ee^{-q^2/2}=0\,.
    \end{equation}
    We hence need to solve for the solution of the differential equation
    \begin{equation}
        p''-2p'q+(2\lambda-1)p=0\,.
    \end{equation}
    
    To solve this type of equations, one can try to expand the solution in an infinite series
    \begin{equation}
        p(q)=\sum_{k=0} a_k q^k\,.
    \end{equation}
    If we substitute this into the Hermite equation, and requiring the coefficients of all powers of \(q\) to be zero, we get the recursion relation
    \begin{equation}
        a_{k+2}=\frac{2k-2n}{(k+2)(k+1)}a_k\,.
    \end{equation}
    First, note that the recursion relation involves two independent sets of coefficients: \(a_k\) with \(k\) even, and \(a_k\) with \(k\) odd. These two sets don't talk to each other and so we have two classes of solutions.
    \begin{align}
        p_{\text{even}}(q)&=a_0\left[1-nq^2+\frac{n(n-2)}{6}q^4+\dots\right] \\
        p_{\text{odd}}(q)&=a_1\left[q-\frac{n-1}{3}q^3+\frac{(n-1)(n-3)}{30}q^5+\dots\right]
    \end{align}

    We may focus on either class of the solution and consider what happens for large \(k\). There are two options: either the recursion relation terminates, so that \(a_k=0\) for all \(k>N\) after some \(N\). Or the recursion relation doesn't terminate and \(a_k\ne 0\) for all \(k\). We're going to argue that only the first option is allowed because if we have infinite terms in the expansion of \(p(q)\), then \(\psi=p(q)\ee^{-q^2/2}\) will be non-normalisable.

    To see this, we observe that if the recursion relation doesn't terminate then, for large \(k\), we have \(a_{k+2}\sim 2a_k/k\). This will give us an exponentially growing function: if we expand
    \begin{equation}
        \ee^{q^2}=\sum_{k=0}^{\infty}\frac{q^{2k}}{k!}=\sum_{k=0}^{\infty}b_k q^k
    \end{equation}
    with
    \begin{equation}
        b_k=\begin{cases}
            \frac{1}{(k/2)!} & \text{if }k\text{ is even} \\
            0 & \text{if }k\text{ is odd,}
        \end{cases}
    \end{equation}
    which also gives \(b_{k+2}=2 b_k/k\) as \(k\to\infty\). Therefore for large \(q\), the wavefunction will scale as
    \begin{equation}
        \psi(q)\sim \ee^{q^2}\ee^{-q^2/2}=\ee^{+q^2/2}\,,
    \end{equation}
    which is clearly not normalisable.

    Therefore, we must require the expansion somehow terminates at some value of \(k\), i.e. \(p(q)\) is a polynomial. This would require \(2k-2n=0\) at some \(k\), meaning that \(n\) is an integer.\footnote{Alternatively, this can be understood by Sturm--Liouville theory. See my notes on IB Mathematical Methods.} The polynomial solution of the \textit{Hermite equation}
    \begin{equation}
        p''-2qp'+2np=0\,,\ n\in\mathbb{Z}
    \end{equation}
    are known as the \textit{Hermite polynomials}, \(H_n\). This gives the wavefunctions and energy spectrum
    \begin{align}
        \psi_n(q)&=H_n(q)\ee^{-\frac{1}{2}q^2}\,,\\
        E_n&=\left(n+\frac{1}{2}\right)\hbar\omega\,.
    \end{align}

    \newpage
    \section{Rigid Rotor}
    Suppose we have two masses \(m_1\) and \(m_2\) connected by a rigid rod of length \(r\) with no external potentials or inter-particle interactions. We will ignore the overall translation of the system and set the centre of mass coordinate to be \(\vb{0}\) (i.e. we are in the centre-of-mass frame). This adds the constraint
    \begin{equation}
        \vb{r}_{\text{CoM}}=\frac{m_1\vb{r}_1+m_2\vb{r}_2}{m_1+m_2}=\vb{0}
    \end{equation}
    We denote the inter-particle vector \(\vb{r}=\vb{r}_2-\vb{r}_1\) with \(\norm{\vb{r}}=\vb{r}\) confined by the length of the rod. It can then be shown that the energy of the particles in this frame is
    \begin{equation}
        H=\frac{1}{2}m_1\dot{\vb{r}}_1^2+\frac{1}{2}m_2\dot{\vb{r}}_2^2=\frac{1}{2}\mu\dot{\vb{r}}^2\,,
    \end{equation}
    where \(\mu=m_1m_2/(m_1+m_2)\) is the \textit{reduced mass} of this system. This shows that we can view the rotation of this rotor as the movement of a single particle of mass \(\mu\) on a sphere of radius \(r\). Defining \(\vb{\omega}=\vb{r}\cross\dot{\vb{r}}/r^2\) the angular velocity and \(I=\mu r^2\) the moment of inertia, this is
    \begin{equation}
        H=\frac{1}{2}I\vb{\omega}^2\,.
    \end{equation} 

    The angular momentum is defined as \(\vb{L}=\vb{r}\cross\vb{p}=I\vb{\omega}\). This allows us to write the classical Hamiltonian as
    \begin{equation}
        H=\frac{\vb{L}^2}{2I}\,.
    \end{equation}
    Quantum Mechanically, this is
    \begin{equation}
        \hat{H}=\frac{\hat{\vb{L}}^2}{2I}\,.
    \end{equation}
    We need to find the eigenvalues and eigenfunctions of \(\hat{\vb{L}}^2\).

    \subsection{Angular Momentum}
    Following the definition \(\hat{\vb{L}}=\hat{\vb{x}}\cross\vb{\vb{p}}\), it is easy to check that the components of \(\hat{\vb{L}}\) are defined by
    \begin{equation}
        \hat{L}_i=-\ii\hbar\sum_{j,k}\varepsilon_{ijk}x_j\pdv{}{x_k}\,,
    \end{equation}
    where \(\epsilon_{ijk}\) is the \textit{Levi-Civita symbol} defined by
    \begin{equation}
        \varepsilon_{ijk}=\begin{cases}
            1 & \text{if }ijk\text{ is an even permutation of }123 \\
            -1 & \text{if }ijk\text{ is an odd permutation of }123 \\
            0 & \text{otherwise.}
        \end{cases}
    \end{equation}
    One can also check the commutator relationship
    \begin{equation}
        [\hat{L}_i,\hat{L}_j]=\ii\hbar\sum_{k}\epsilon_{ijk} \hat{L}_k\,.
    \end{equation}
    This is not zero, so one cannot find a state that is simultaneously the eigenstates of two components of angular momentum.
    
    The squared magnitude of angular momentum can be calculated by \(\hat{\vb{L}}^2=\sum_i \hat{L}_i^2\). The expression is simpler if we switch to the spherical polar coordinate
    \begin{equation}\label{J_squared}
        \hat{\vb{L}}^2=-\hbar^2\left[\frac{1}{\sin\theta}\pdv{}{\theta}\left(\sin\theta\pdv{}{\theta}\right) + \frac{1}{\sin^2\theta}\pdv[2]{}{\varphi}\right]\,.
    \end{equation}
    We can also check the commutator
    \begin{equation}
        [\hat{\vb{L}}^2,\hat{L}_i]=0\,,
    \end{equation}
    so one can work out the simultaneous eigenstate of \(\hat{\vb{L}}^2\) and one component of \(\vb{L}\) --- it is conventional to choose \(\hat{L}_z\) due to its simple form in spherical polar coordinate:
    \begin{equation}
        \hat{L}_z=-\ii\hbar\pdv{}{\varphi}\,.
    \end{equation}

    We will first work out the eigenfunction of \(\hat{L}_z\). The eigenvalue equation \(\hat{L}_z\ket{\lambda}=\lambda\ket{\lambda}\) in spherical polar coordinate is
    \begin{equation}
        -\ii\hbar\pdv{\Phi(\varphi)}{\varphi}=\lambda\Phi(\varphi)\,.
    \end{equation}
    This has solution
    \begin{equation}
        \Phi(\varphi)=\ee^{\ii \lambda\varphi/\hbar}\,,
    \end{equation}
    while the single-valuedness of the wavefunction requires \(\Phi(\varphi+2\pi)=\Phi(\varphi)\), so
    \begin{equation}
        \frac{\ii \lambda(\varphi+2\pi)}{\hbar}=\frac{\ii \lambda\varphi}{\hbar}+2m\pi
    \end{equation}
    for \(m\in\mathbb{Z}\), which simplifies to \(\lambda=m\hbar\). We again see that the boundary condition requires \(\hat{L}_z\) to have a discrete spectrum. Labelling the eigenstates by \(\ket{m}\), we have eigenfunctions and spectrum
    \begin{align}
        \Phi_m(\varphi)\equiv\braket{\varphi}{m}&=\ee^{im\varphi}\\
        \hat{L}_z\ket{m}&=\hbar m\ket{m}
    \end{align}
    for \(m\in\mathbb{Z}\).

    Next we find the eigenfunction of the angular momentum squared operator (\ref{J_squared}). Since \(\hat{\vb{L}}^2\) commutes with \(\hat{L}_z\), we can find the simultaneous eigenfunction of them. Therefore we can label a state by \(\ket{\chi,m}\), with eigenvalue equations
    \begin{align}
        \hat{L}_z\ket{\chi,m}&=\hbar m\ket{\chi,m}\\
        \hat{\vb{L}}^2\ket{\chi,m}&=\chi\ket{\chi,m}\,.
    \end{align}
    Working in the spherical polar coordinates, we must have
    \begin{align}
        \braket{\theta,\varphi}{\chi,m}=\Theta_{\chi}(\theta)\Phi_m(\varphi)
    \end{align}
    with \(\Phi_m(\varphi)=\ee^{\ii m\varphi}\) for some function \(\Theta_{\chi}(\theta)\).
    If we substitute this into the eigenvalue equation in spherical polar coordinate, we get
    \begin{equation}\label{associated_Legendre_eqn}
        -\frac{\hbar^2}{\sin^2\theta}\left[\sin\theta\pdv{}{\theta}\left(\sin\theta\pdv{}{\theta}\right)-m^2\right]\Theta(\theta)=\chi \Theta(\theta)\,.
    \end{equation}
    Solutions to this equation are a well-studied class of mathematical functions called \textit{associated Legendre Polynomials}. We are not solving them right here, but the essence is the same as how we solved the Hermite equation --- just with an extra first step of substitution \(x=\cos\theta\). We will directly state the solution here. Their construction comes in two steps. First, we introduce an associated set of functions known as \textit{(ordinary) Legendre Polynomials}, \(P_\ell(x)\). These obey the differential equation
    \begin{equation}
        \dv{}{x}\left[(1-x^2)\dv{P_\ell}{x}\right]+\ell(\ell+1)P_\ell(x)=0\,,\ \ell\in\mathbb{N}_0\,.
    \end{equation}
    The \(P_\ell(x)\), with \(\ell\in\mathbb{N}_0\), are polynomials of degree \(\ell\). The eigenfunction solutions (\ref{associated_Legendre_eqn}) are then \textit{associated Legendre polynomials} \(P_{\ell,m}(\cos\theta)\), defined by
    \begin{equation}
        \Theta(\theta)=P_{\ell,m}(\cos\theta)=(\sin\theta)^{\abs{m}}\dv[\abs{m}]{}{(\cos\theta)}P_\ell(\cos\theta)\,.
    \end{equation}
    The corresponding eigenvalue is
    \begin{equation}
        \chi=\ell(\ell+1)\hbar^2\,.
    \end{equation}
    Because \(P_\ell(\cos\theta)\) is a polynomial of degree \(\ell\), you only get to differentiate it \(\ell\) times before it vanishes. This means that the \(\hat{L}_z\) angular momentum is constrained to take values that lie in the range
    \begin{equation}
        -\ell\le m\le \ell\,.
    \end{equation}

    The conclusion of the above analysis is that the simultaneous eigenstates of \(\hat{\vb{L}}^2\) and \(\hat{L}_z\) are given by quantum state labelled by two integers \(\ell\) and \(m\), with \(\abs{m}\le\ell\)
    \begin{equation}
        \braket{\theta,\varphi}{\ell,m}=Y_{\ell,m}(\theta,\varphi)=P_{\ell,m}(\cos\theta)e^{im\varphi}
    \end{equation}
    known as the \textit{spherical harmonics}. The corresponding eigenvalues are
    \begin{align*}
        \hat{\vb{L}}^2Y_{\ell,m}(\theta,\varphi)&=\ell(\ell+1)\hbar^2 Y_{\ell,m}(\theta,\varphi)\,,\\
        \hat{L}_z Y_{\ell,m}(\theta,\varphi)&=m\hbar Y_{\ell,m}(\theta,\varphi)\,.
    \end{align*}
    The integer \(\ell\) is called the \textit{total angular momentum quantum number} (even though, strictly speaking, the total angular momentum is really \(\ell(\ell+1)\hbar\).) The integer \(m\) is called the \textit{azimuthal angular momentum}. For each value of \(\ell\), there are \(2\ell+1\) values \(m\) can take.

    Back to the rigid rotor, the eigenstates can also be labelled by \(\ket{\ell,m}\). The eigenvalues are
    \begin{equation}
        \hat{H}\ket{\ell,m}=\frac{\hbar^2}{2I}\ell(\ell+1)\ket{\ell,m}\,.
    \end{equation}
    The states with the same \(\ell\) but different \(m\) are therefore degenerate, with degeneracy \(2\ell+1\).

    \newpage
    \section{Hydrogen Atom}
    We will consider the hydrogen atom as an electron of mass \(m\) moving in the electric field generated by a fixed proton. This is a pretty good approximation since the proton is approximately \(1836\) times as heavy as the electron, so we can consider the proton to be essentially stationary --- but we shall see that this will come and bite us in the chapter of Relativistic Effects when we are trying to make really accurate spectroscopic measurement.\footnote{Long story short: we should really use the reduced mass of the proton-electron system.}

    The electronic Hamiltonian is therefore
    \begin{equation}
        \hat{H}=-\frac{\hbar^2}{2m_e}\laplacian-\frac{e^2}{4\pi\epsilon_0 r}\,.
    \end{equation}
    To respect the spherical symmetry of the system, we should use the spherical polar coordinate system, in which the Laplacian is
    \begin{equation}
        \laplacian=\frac{1}{r^2}\pdv{}{r}\left(r^2\pdv{}{r}\right)+\frac{1}{r^2\sin\theta}\pdv{}{\theta}\left(\sin\theta\pdv{}{\theta}\right)+\frac{1}{r^2\sin^2\theta}\pdv[2]{}{\varphi}\,.
    \end{equation}
    We see a familiar fragment in the angular part --- the squared modulus of the angular momentum operator:
    \begin{equation}
        \laplacian=\frac{1}{r^2}\left[\pdv{}{r}\left(r^2\pdv{}{r}\right)-\frac{1}{\hbar^2}\hat{\vb{L}}^2\right]\,.
    \end{equation}
    The full Hamiltonian is therefore
    \begin{equation}
        \hat{H}=\frac{1}{r^2}\left[\frac{1}{2m_e}\hat{\vb{L}}^2-\frac{\hbar^2}{2m_e}\pdv{}{r}\left(r^2\pdv{}{r}\right)-\frac{e^2}{4\pi\epsilon_0}r\right]\,.
    \end{equation}
    This inspires us to try solution of the form \(R(r)Y_{\ell,m}(\theta,\varphi)\). Substituting this into the time-independent Schr\"{o}dinger equation, we get
    \begin{align}
        \hat{H}R(r)Y_{\ell,m}&=\frac{1}{r^2}\left[\frac{1}{2m_e}R(r)\hat{\vb{L}}^2 Y_{\ell,m}-Y_{\ell,m}\frac{\hbar^2}{2m_e}\pdv{}{r}\left(r^2\pdv{R}{r}\right)-\frac{e^2}{4\pi\epsilon_0}rR(r)Y_{\ell,m}\right] \notag \\
        &=\frac{1}{r^2}\left[\frac{\hbar^2\ell(\ell+1)}{2m_e}R(r)-\frac{\hbar^2}{2m_e}\pdv{}{r}\left(r^2\pdv{R}{r}\right)-\frac{e^2}{4\pi\epsilon_0}rR(r)\right]Y_{\ell,m}=ER(r)Y_{\ell,m}\,.
    \end{align}
    This reduces to the eigenvalue equation
    \begin{equation}
        -\frac{\hbar^2}{2m_e}\left(\dv[2]{R}{r}+\frac{2}{r}\dv{R}{r}\right)+\left(-\frac{e^2}{4\pi\epsilon_0 r}+\frac{\hbar^2\ell(\ell+1)}{2m_e r^2}\right)R=ER\,.
    \end{equation}
    Before going any further, we would like to define the following quantities to reduce the constants we have to write:
    \begin{equation}
        \frac{1}{a^2}=-\frac{2m_e E}{\hbar^2}\,,\ \beta=\frac{e^2 m_e}{2\pi\epsilon_0\hbar^2}\,.
    \end{equation}
    You can check that \(a\) has the dimension of length. This cleans up our equation a little:
    \begin{equation}\label{hydrogen_R_eqn}
        \dv[2]{R}{r}+\frac{2}{r}\dv{R}{r}-\frac{\ell(\ell+1)}{r^2}R+\frac{\beta}{r}R=\frac{R}{a^2}\,.
    \end{equation}
    All we have to do now is solve it.

    We need some inspiration on what form of solution to try. First, let's look at the asymptotic behaviour of solutions as \(r\to\infty\). Here the eigenvalue equation (\ref{hydrogen_R_eqn}) is dominated by
    \begin{equation}\label{hydrogenasympatinfty}
         \dv[2]{R}{r}\approx\frac{R}{a^2}\,.
    \end{equation}
    This is solved by \(R(r)=\ee^{\pm r/a}\), and we clearly have to discard the \(e^{+r/a}\) solution because it is not normalisable. So, we must have \(R(r)\sim e^{\pm r/a}\) as \(r\to\infty\). We see that \(a\), which recall is related to the inverse energy of the system, sets the characteristic size of the wavefunction. Meanwhile, near the origin \(r=0\) the dominant terms are
   \begin{equation}
        \dv[2]{R}{r}+\frac{2}{r}\dv{R}{r}-\frac{\ell(\ell+1)}{r^2}R\approx 0\text{ for }r\ll 1\,.
    \end{equation}
    If we make the power-law ansatz \(R\sim r^\alpha\), we find
   \begin{equation}
        \alpha(\alpha-1)+2\alpha-\ell(\ell+1)=0\,,
    \end{equation}
    which has two solutions: \(\alpha=\ell\) and \(\alpha=-(\ell+1)\). We have to discarded the second solution because it results in a wavefunction that diverges at the origin, and so \(R(r)\sim r^\ell\) as \(r\to 0\).

    All of this motivates us to look for solutions of the form
    \begin{equation}\label{hydrogen_radial_ansatz}
        R(r)=r^\ell f(r)e^{-r/a}\,,
    \end{equation}
    where \(f(r)\) is a polynomial
   \begin{equation}
        f(r)=\sum_{k=0}^{\infty}c_kr^k\,,
    \end{equation}
    where we must have \(c_0\ne 0\) so that we get the right behaviour at small \(r\). We can now substitute the ansatz (\ref{hydrogen_radial_ansatz}) into the equation (\ref{hydrogen_R_eqn}), which gives
   \begin{equation}
        \dv[2]{f}{r}+2\left(\frac{\ell+1}{r}-\frac{1}{a}\right)\dv{f}{r}-\frac{1}{ar}\left(2(\ell+1)-\beta a\right)f=0\,.
    \end{equation}
    Next, we substitute the power-law expansion of \(f\) into this differential equation to find
   \begin{equation}
        \sum_{k=0}^{\infty}c_k\left[(k(k-1)+2k(\ell+1))r^{k-2}-\frac{1}{a}(2k+2(\ell+1)-\beta a)r^{k-1}\right]=0\,,
    \end{equation}
    which gives the recursion relation
    \begin{equation}\label{hydrogen_recursion}
        c_k=\frac{1}{ak}\frac{2(k+l)-\beta a}{k+2\ell+1}c_{k-1}\,.
    \end{equation}

    Similar to the argument we made when discussing the harmonic oscillator, for a wavefunction to be normalisable, this recursion relation must terminate at some time --- otherwise this brings us back to the \(R(r)\sim \ee^{+r/a}\) situation.

    This means that there should be a positive integer \(q\) for which \(c_q=0\), while \(c_k\ne 0\) for all \(k<q\). Clearly this holds only if \(a\) takes special values,
   \begin{equation}
        a=\frac{2}{\beta}(q+\ell)\,,\; q=1,2,\dots\,.
    \end{equation}
    Alternatively, since both \(q\) and \(\ell\) are integers, we usually define the integer
   \begin{equation}
        n=q+\ell\,,
    \end{equation}
    which obviously obeys \(n>\ell\). We then have
   \begin{equation}
        a=\frac{2}{\beta}n\,.
    \end{equation}
    This then gives
    \begin{equation}\label{hydrogenenergy}
        E=-\frac{e^4m_e}{32\pi^2\epsilon_0^2\hbar^2}\frac{1}{n^2}
    \end{equation}
    for \(n\in\mathbb{N}\). This is our final result for the energy spectrum of the hydrogen atom. The integer \(n\) is called the \textit{principal quantum number}. The full (bound) state of the hydrogen atom can therefore be labelled by three numbers: \(\ket{n,\ell,m}\), with
    \begin{align}
        \hat{H}\ket{n,\ell,m}&=-\frac{e^4m_e}{32\pi^2\epsilon_0^2\hbar^2}\frac{1}{n^2}\ket{n,\ell,m} & &n\in\mathbb{N} \\
        \hat{\vb{L}}^2\ket{n,\ell,m}&=\hbar^2\ell(\ell+1)\ket{n,\ell,m} & &0\le\ell<n \\
        \hat{L}_z\ket{n,\ell,m}&=\hbar m\ket{n,\ell,m} & &-\ell\le m\le \ell\,.
    \end{align}

    It is common to adopt the atomic units, in which \(\hbar=4\pi\epsilon_0=m_e=e=1\). The corresponding unit of length is \textit{Bohr radius}
    \begin{equation}
        a_0=\frac{4\pi\epsilon_0\hbar^2}{e^2 m_e}=1
    \end{equation}
    and unit of energy is Hartree
    \begin{equation}
        E_H=\frac{\hbar^2}{m_e a_0^2}=\frac{e^4 m_e}{16\pi^2\epsilon_0^2\hbar^2}=1\,.
    \end{equation}
    Therefore, in atomic unit, the hydrogen energy level is simply \(-\frac{1}{2n^2}\).

    There's actually an extra surprise. The energy spectrum of hydrogen does not depend on the angular momentum \(\ell\). This is not generally true for any system --- in fact, the are only two 3D systems where the energy spectrum doesn't explicitly depend on angular momentum are the harmonic oscillator and the hydrogen atom!

    There is no \(\hat{L}_z\) involved in the expression of the Hamiltonian, so we are quite happy to accept that the energy of the hydrogen is independent of \(m\), i.e. the states that differ only by \(m\) are degenerate. This naturally follows from rotational symmetry of hydrogen atom --- it should have no preference on in which direction the angular momentum lies. However, we do have \(\hat{L}^2\) appearing explicitly in the Hamiltonian, so we generally would not expect states with different \(\ell\)'s to be degenerate. In fact, this is true for any other atom, or indeed for any other central potentials --- the degeneracy is \(2\ell+1\) for the states that differ by \(m_\ell\) only. However, there is another conserved quantity in hydrogen atom that arises because the potential is exactly \(\sim r^{-1}\). It is called the \textit{Runge--Lenz} vector, which also appears in many other places like planetary orbits. This enhances the symmetry of a hydrogen atom. Roughly speaking, aside from simply rotating our system, we can also trade radial kinetic energy and Coulomb potential for different orbital kinetic energy whilst keeping the total energy constant. This means that the degeneracy at level of the state at level \(n\) is
   \begin{equation}
        \text{Total Degeneracy}=\sum_{\ell=0}^{n-1}(2\ell+1)=n^2\,.
    \end{equation}

    \subsubsection{The Wavefunctions}
    The recursion relation (\ref{hydrogen_recursion}) allows us to easily construct the corresponding wavefunctions. For a state \(\ket{n,\ell,m}\),
   \begin{equation}
        \psi_{n,\ell,m}(r,\theta,\phi)=r^\ell f_{n,\ell}(r)e^{-r/a}Y_{\ell,m}(\theta,\phi)\,,
    \end{equation}
    where \(a=na_0=2n/\beta\) and \(f_{n,\ell}\) is a polynomial of degree \(n-\ell-1\), defined by
   \begin{equation}
        f_{n,\ell}(r)=\sum_{k=0}^{n-\ell-1}c_k r^k\,,
    \end{equation}
    with
   \begin{equation}
        c_k=\frac{2}{ak}\frac{k+\ell-n}{k+2\ell+1}c_{k-1}\,.
    \end{equation}
    These are known as the \textit{generalised Laguerre polynomials}. They are more conventionally written as \(f_{n,\ell}\coloneqq L_{n-\ell-1}^{2\ell+1}(2r/na_0)\).

    The \(n=1\) ground state wavefunction necessarily has vanishing angular momentum. It is simply
   \begin{equation}
        \psi{100}(r)=\sqrt{\frac{1}{\pi a_0^3}}e^{-r/a_0}\,,
    \end{equation}
    where the coefficient in front is chosen to normalise the wavefunction. We now see clearly that the Bohr radius \(a_0\) set the size of the ground state of the hydrogen atom.

    \begin{figure}
        \centering
        \tikz{
            \draw[->] (-0.2,0)--(4,0) node[above]{\(r\)};
            \draw[->] (0,-1)--(0,2.5) node[right]{\(\psi_{1,0}(r)\)};
            \draw (0.6,0.1)--(0.6,-0.1) node[below]{\(a_0\)};
            \draw[domain=0:4, smooth, variable=\x,samples=100] plot ({\x}, {2*2.72^(-\x/0.6)});
        }
        \tikz{
            \draw[->] (-0.2,0)--(4,0) node[above]{\(r\)};
            \draw[->] (0,-1)--(0,2.5) node[right]{\(\psi_{2,0}(r)\)};
            \draw (0.3,0.1)--(0.3,-0.1) node[below]{\(a_0\)};
            \draw[domain=0:4, smooth, variable=\x,samples=100] plot ({\x}, {(2-\x/0.3)*2.72^(-\x/0.6)});
        }
        \tikz{
            \draw[->] (-0.2,0)--(4,0) node[above]{\(r\)};
            \draw[->] (0,-1)--(0,2.5) node[right]{\(\psi_{3,0}(r)\)};
            \draw (0.2,0.1)--(0.2,-0.1) node[below]{\(a_0\)};
            \draw[domain=0:4, smooth, variable=\x,samples=100] plot ({\x}, {(2-1.33*\x/0.2+3.704*\x*\x)*2.72^(-\x/0.6)});
        }
        \caption{The (un-normalised) s radial wavefunctions for \(n= 1,2\) and \(3\).}
        \label{Fig:hydrogen_s}
    \end{figure}

    To get some sense for the higher wavefunctions, we've plotted the \(s\) radial wavefunctions for \(n=1\) to \(n=3\) in \cref{Fig:hydrogen_s}. Note that each successive higher energy state contains an additional \(\psi=0\) node. In \cref{Fig:hydrogen_spd} we've fixed the energy level to \(n=3\) and plotted successive higher angular momentum modes. This time we lose a node each time that \(\ell\) increases.

    \begin{figure}
        \centering
        \tikz{
            \draw[->] (-0.2,0)--(4,0) node[above]{\(r\)};
            \draw[->] (0,-1)--(0,2.5) node[right]{\(\psi_{3,0}(r)\)};
            \draw (0.2,0.1)--(0.2,-0.1) node[below]{\(a_0\)};
            \draw[domain=0:4, smooth, variable=\x,samples=100] plot ({\x}, {(2-1.33*\x/0.2+3.704*\x*\x)*2.72^(-\x/0.6)});
        }
        \tikz{
            \draw[->] (-0.2,0)--(4,0) node[above]{\(x\)};
            \draw[->] (0,-1)--(0,2.5) node[right]{\(\psi_{3,1}(r)\)};
            \draw (0.2,0.1)--(0.2,-0.1) node[below]{\(a_0\)};
            \draw[domain=0:4, smooth, variable=\x,samples=100] plot ({\x}, {(\x/0.2)*(2-\x/0.6)*2.72^(-\x/0.6)});
        }
        \tikz{
            \draw[->] (-0.2,0)--(4,0) node[above]{\(x\)};
            \draw[->] (0,-1)--(0,2.5) node[right]{\(\psi_{3,2}(r)\)};
            \draw (0.2,0.1)--(0.2,-0.1) node[below]{\(a_0\)};
            \draw[domain=0:4, smooth, variable=\x,samples=100] plot ({\x}, {(\x*\x/0.2)*2.72^(-\x/0.6)});
        }
        \caption{The (un-normalised) radial wavefunctions for fixed \(n=3\) with angular momentum varying from \(\ell=0\) (s) to \(\ell=1\) (p) to \(\ell=2\) (d).}
        \label{Fig:hydrogen_spd}
    \end{figure}

    For any wavefunction in the \(n^\text{th}\) energy level, the peak can be approximated by writing \(\psi(r)\approx r^{n-1}e^{-r/na_0}\), which corresponds to a probability distribution \(P(r)\approx r^{2(n-1)}e^{-2r/na_0}\). This has a maximum when \(P'(r)=0\), or
   \begin{equation}
        \frac{2(n-1)}{r}\approx\frac{2}{na_0}\quad\implies\quad r\approx n(n-1)a_0\,.
    \end{equation}
    We see that the spatial extent of the higher energy states grows roughly as \(n^2\).


\end{document}